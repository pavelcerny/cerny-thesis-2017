%% History:
% Pavel Tvrdik (26.12.2004)
%  + initial version for PhD Report
%
% Daniel Sykora (27.01.2005)
%
% Michal Valenta (3.12.2008)
% rada zmen ve formatovani (diky M. Duškovi, J. Holubovi a J. Žďárkovi)
% sjednoceni zdrojoveho kodu pro anglickou, ceskou, bakalarskou a diplomovou praci

% One-page layout: (proof-)reading on display
\documentclass[11pt,oneside,a4paper]{book}
% Two-page layout: final printing
%\documentclass[11pt,twoside,a4paper]{book}   
%=-=-=-=-=-=-=-=-=-=-=-=--=%
% The user of this template may find useful to have an alternative to these 
% officially suggested packages:
\usepackage[english]{babel}
\usepackage[T1]{fontenc} % pouzije EC fonty 
% pripadne pisete-li cesky, pak lze zkusit take:
% \usepackage[OT1]{fontenc} 
\usepackage[utf8]{inputenc}




%=-=-=-=-=-=-=-=-=-=-=-=--=%
% In case of problems with PDF fonts, one may try to uncomment this line:
%\usepackage{lmodern}
%=-=-=-=-=-=-=-=-=-=-=-=--=%
%=-=-=-=-=-=-=-=-=-=-=-=--=%
% Depending on your particular TeX distribution and version of conversion tools 
% (dvips/dvipdf/ps2pdf), some (advanced | desperate) users may prefer to use 
% different settings.
% Please uncomment the following style and use your CSLaTeX (cslatex/pdfcslatex) 
% to process your work. Note however, this file is in UTF-8 and a conversion to 
% your native encoding may be required. Some settings below depend on babel 
% macros and should also be modified. See \selectlanguage \iflanguage.
%\usepackage{czech}  %%%%%\usepackage[T1]{czech} %%%%[IL2] [T1] [OT1]
%=-=-=-=-=-=-=-=-=-=-=-=--=%

% MY ADDED PACKAGES
\usepackage{booktabs}
\usepackage[table,xcdraw]{xcolor}
\usepackage{url}
\usepackage{gensymb}

\usepackage[usenames,dvipsnames]{pstricks}
\usepackage{epsfig}
% \usepackage{pst-grad} % For gradients
% \usepackage{pst-plot} % For axes
\usepackage[space]{grffile} % For spaces in paths
\usepackage{etoolbox} % For spaces in paths
\makeatletter % For spaces in paths



%%%%%%%%%%%%%%%%%%%%%%%%%%%%%%%%%%%%%%%
% Styles required in your work follow %
%%%%%%%%%%%%%%%%%%%%%%%%%%%%%%%%%%%%%%%
\usepackage{graphicx}
\usepackage{caption}
\usepackage{subcaption}

\usepackage{scrextend}
\addtokomafont{labelinglabel}{\sffamily}
%\usepackage{indentfirst} %1. odstavec jako v cestine.

\usepackage{k336_thesis_macros} % specialni makra pro formatovani DP a BP
% muzete si vytvorit i sva vlastni v souboru k336_thesis_macros.sty
% najdete  radu jednoduchych definic, ktere zde ani nejsou pouzity
% napriklad: 
% \newcommand{\bfig}{\begin{figure}\begin{center}}
% \newcommand{\efig}{\end{center}\end{figure}}
% umoznuje pouzit prikaz \bfig namisto \begin{figure}\begin{center} atd.


%%%%%%%%%%%%%%%%%%%%%%%%%%%%%%%%%%%%%
% Zvolte jednu z moznosti 
% Choose one of the following options
%%%%%%%%%%%%%%%%%%%%%%%%%%%%%%%%%%%%%
% \newcommand\TypeOfWork{Diplomová práce} \typeout{Diplomova prace}
\newcommand\TypeOfWork{Master's Thesis}   \typeout{Master's Thesis} 
% \newcommand\TypeOfWork{Bakalářská práce}  \typeout{Bakalarska prace}
% \newcommand\TypeOfWork{Bachelor's Project}  \typeout{Bachelor's Project}


%%%%%%%%%%%%%%%%%%%%%%%%%%%%%%%%%%%%%
% Zvolte jednu z moznosti 
% Choose one of the following options
%%%%%%%%%%%%%%%%%%%%%%%%%%%%%%%%%%%%%
% nabidky jsou z: http://www.fel.cvut.cz/cz/education/bk/prehled.html

%\newcommand\StudProgram{Elektrotechnika a informatika, dobíhající, Bakalářský}
%\newcommand\StudProgram{Elektrotechnika a informatika, dobíhající, Magisterský}
% \newcommand\StudProgram{Elektrotechnika a informatika, strukturovaný, Bakalářský}
\newcommand\StudProgram{Open Informatics, Master degree}
% \newcommand\StudProgram{Softwarové technologie a management, Bakalářský}
% English study:
% \newcommand\StudProgram{Electrical Engineering and Information Technology}  % bachelor programe
% \newcommand\StudProgram{Electrical Engineering and Information Technology}  %master program


%%%%%%%%%%%%%%%%%%%%%%%%%%%%%%%%%%%%%
% Zvolte jednu z moznosti 
% Choose one of the following options
%%%%%%%%%%%%%%%%%%%%%%%%%%%%%%%%%%%%%
% nabidky jsou z: http://www.fel.cvut.cz/cz/education/bk/prehled.html

%\newcommand\StudBranch{Výpočetní technika}   % pro program EaI bak. (dobihajici i strukt.)
\newcommand\StudBranch{Software engineering}   % pro prgoram EaI mag. (dobihajici i strukt.)
%\newcommand\StudBranch{Softwarové inženýrství}            %pro STM
%\newcommand\StudBranch{Web a multimedia}                  % pro STM
%\newcommand\StudBranch{Computer Engineering}              % bachelor programe
%\newcommand\StudBranch{Computer Science and Engineering}  % master programe


%%%%%%%%%%%%%%%%%%%%%%%%%%%%%%%%%%%%%%%%%%%%
% Vyplnte nazev prace, autora a vedouciho
% Set up Work Title, Author and Supervisor
%%%%%%%%%%%%%%%%%%%%%%%%%%%%%%%%%%%%%%%%%%%%

\newcommand\WorkTitle{Localization of visually impaired pedestrians by means of a dialog system}
\newcommand\FirstandFamilyName{Bc. Pavel Cerny}
\newcommand\Supervisor{Ing. Jan Balata}


% Pouzijete-li pdflatex, tak je prijemne, kdyz bude mit vase prace
% funkcni odkazy i v pdf formatu
\usepackage[
pdftitle={\WorkTitle},
pdfauthor={\FirstandFamilyName},
bookmarks=true,
colorlinks=true,
breaklinks=true,
urlcolor=red,
citecolor=blue,
linkcolor=blue,
unicode=true,
]
{hyperref}




\begin{document}
	
	%%%%%%%%%%%%%%%%%%%%%%%%%%%%%%%%%%%%%
	% Zvolte jednu z moznosti 
	% Choose one of the following options
	%%%%%%%%%%%%%%%%%%%%%%%%%%%%%%%%%%%%%
	%\selectlanguage{czech}
	\selectlanguage{english} 
	
	% prikaz \typeout vypise vyse uvedena nastaveni v prikazovem okne
	% pro pohodlne ladeni prace
	
	
	\iflanguage{czech}{
		\typeout{************************************************}
		\typeout{Zvoleny jazyk: cestina}
		\typeout{Typ prace: \TypeOfWork}
		\typeout{Studijni program: \StudProgram}
		\typeout{Obor: \StudBranch}
		\typeout{Jmeno: \FirstandFamilyName}
		\typeout{Nazev prace: \WorkTitle}
		\typeout{Vedouci prace: \Supervisor}
		\typeout{***************************************************}
		\newcommand\Department{Katedra počítačů}
		\newcommand\Faculty{Fakulta elektrotechnická}
		\newcommand\University{České vysoké učení technické v Praze}
		\newcommand\labelSupervisor{Vedoucí práce}
		\newcommand\labelStudProgram{Studijní program}
		\newcommand\labelStudBranch{Obor}
	}{
		\typeout{************************************************}
		\typeout{Language: english}
		\typeout{Type of Work: \TypeOfWork}
		\typeout{Study Program: \StudProgram}
		\typeout{Study Branch: \StudBranch}
		\typeout{Author: \FirstandFamilyName}
		\typeout{Title: \WorkTitle}
		\typeout{Supervisor: \Supervisor}
		\typeout{***************************************************}
		\newcommand\Department{Department of Computer Science and Engineering}
		\newcommand\Faculty{Faculty of Electrical Engineering}
		\newcommand\University{Czech Technical University in Prague}
		\newcommand\labelSupervisor{Supervisor}
		\newcommand\labelStudProgram{Study Program} 
		\newcommand\labelStudBranch{Field of Study}
	}
	
	
	%%%%%%%%%%%%%%%%%%%%%%%%%%%%%%%%%%%%%%%%
	% MY SCRIPTS
	\newcommand\poi{Naviterier - POI}
	\newcommand\gps{Naviterier - GPS \& compass}
	\newcommand\reversegeo{Naviterier - Reverse Geocoding}
	
	\newcommand{\qo}[1]{\emph{#1}}
	
	
	
	%	%%%%%%%%%%%%%%%%%%%%%%%%%%    Poznamky ke kompletaci prace
	%	% Nasledujici pasaz uzavrenou v {} ve sve praci samozrejme 
	%	% zakomentujte nebo odstrante. 
	%	% Ve vysledne svazane praci bude nahrazena skutecnym 
	%	% oficialnim zadanim vasi prace.
	%	{
	%		\pagenumbering{roman} \cleardoublepage \thispagestyle{empty}
	%		\chapter*{Na tomto místě bude oficiální zadání vaší práce}
	%		\begin{itemize}
	%			\item Toto zadání je podepsané děkanem a vedoucím katedry,
	%			\item musíte si ho vyzvednout na studiijním oddělení Katedry počítačů na Karlově náměstí,
	%			\item v jedné odevzdané práci bude originál tohoto zadání (originál zůstává po obhajobě na katedře),
	%			\item ve druhé bude na stejném místě neověřená kopie tohoto dokumentu (tato se vám vrátí po obhajobě).
	%		\end{itemize}
	%		\newpage
	%	}
	%	
	%	%%%%%%%%%%%%%%%%%%%%%%%%%%    Titulni stranka / Title page 
	%	
	%	\coverpagestarts
	%	
	%	%%%%%%%%%%%%%%%%%%%%%%%%%%%    Podekovani / Acknowledgements 
	%	
	%	%\acknowledgements
	%	%\noindent
	%	%Zde můžete napsat své poděkování, pokud chcete a máte komu děkovat.
	%	
	%	
	%	%%%%%%%%%%%%%%%%%%%%%%%%%%%   Prohlaseni / Declaration 
	%	
	%	\declaration{In~Prague on May 26, 2017}
	%	%\declaration{In Kořenovice nad Bečvárkou on May 15, 2008}
	%	
	%	
	%	%%%%%%%%%%%%%%%%%%%%%%%%%%%%    Abstract 
	%	
	%	\abstractpage
	%	
	%	Translation of Czech abstract into English.
	%	
	%	% Prace v cestine musi krome abstraktu v anglictine obsahovat i
	%	% abstrakt v cestine.
	%	\vglue60mm
	%	
	%	\noindent{\Huge \textbf{Abstrakt}}
	%	\vskip 2.75\baselineskip
	%	
	%	\noindent
	%	Abstrakt práce by měl velmi stručně vystihovat její podstatu. Tedy čím se práce zabývá a co je jejím výsledkem/přínosem.
	%	
	%	\noindent
	%	Očekávají se cca 1 -- 2 odstavce, maximálně půl stránky.
	%	
	%	%%%%%%%%%%%%%%%%%%%%%%%%%%%%%%%%  Obsah / Table of Contents 
	%	
	%	\tableofcontents
	%	
	%	
	%	%%%%%%%%%%%%%%%%%%%%%%%%%%%%%%%  Seznam obrazku / List of Figures 
	%	
	%	\listoffigures
	%	
	%	
	%	%%%%%%%%%%%%%%%%%%%%%%%%%%%%%%%  Seznam tabulek / List of Tables
	%	
	%	\listoftables
	
	
	%**************************************************************
	
	\mainbodystarts
	% horizontalní mezera mezi dvema odstavci
	%\parskip=5pt
	%11.12.2008 parskip + tolerance
	\normalfont
	\parskip=0.2\baselineskip plus 0.2\baselineskip minus 0.1\baselineskip
	
	% Odsazeni prvniho radku odstavce resi class book (neaplikuje se na prvni 
	% odstavce kapitol, sekci, podsekci atd.) Viz usepackage{indentfirst}.
	% Chcete-li selektivne zamezit odsazeni 1. radku nektereho odstavce,
	% pouzijte prikaz \noindent.
	
	%**************************************************************
	
	% Pro snadnejsi praci s vetsimi texty je rozumne tyto rozdelit
	% do samostatnych souboru nejlepe dle kapitol a tyto potom vkladat
	% pomoci prikazu \include{jmeno_souboru.tex} nebo \include{jmeno_souboru}.
	% Napr.:
	% \include{1_uvod}
	% \include{2_teorie}
	% atd...
	%*****************************************************************************
	TAGLINE - jakou metodu vybrat pro lokalizaci slepce, když chce pak někam dojít
	\chapter{Analysis}
	They say, the GPS system is extremely precise. They say the system is precise up to 1 meter. That's true just outside the city in areas with clean view to the sky. In center of Prague (Czech Republic). It's not so perfect. The city area, covered with houses and narrow streets. The precission differs.
	This is the recording of a GPS:
	%image A showing walked history
	
	CAPTION A - this is what was recorded during the walk
	% image B showing real path
	CAPTION B - ...and this is the path the user really took.
	
	There is a siginificant difference. And we can clearly say the GPS in Prague is not sufficient to determine which side of the street we are standing on.
	%image demonstrate what is left and what right sidewalk
	 And some times even can't say, what street are we standing in.
	% image gps in the middle of a house
	
	\section{GPS accuracy}
	I performed a test in order to know how acurate is the GPS in the city. I used two phones to collect the data. Both of them were running app GPS logger.
	
	phone \#1  BlackBerry Q10
	phone \#2  Android - HTC Desire X
	
	app		GPS logger \ref{gps-logger}
	
	I walked 1.1km long path arround Charles Square and nearby surroudings. The app logged the GPS postion on both phones every 60 seconds of walk. This way I collected 23 recordings (totall). Then I compared these recordings with real positions on the map. And I computed the difference between estiamted accuracy and real accuracy.
	
	%img tables - přesnost GPS z SVP
	
	The results showed two things:
	\begin{enumerate}
		\item the real accuracy is nearly allways $\leq$ estimated accuracy
		\item the average estimated accuracy is mostly allways $\leq 50m$
	\end{enumerate}
	
	
	
	
	\section{Area of Interest}
	
	\section{Naviterier}
	
	\section{Goals}
		\begin{itemize}
			\item dialog application
			\item without GPS
			\item \emph{ (or as minimally as possible)}
		\end{itemize}
		\chapter{Development}
		The design process and validating through user testing made me to develop and test 7 methods. These methods are:
		\begin{multicols}{2}
			\begin{itemize}
				\item Describe the surroundings
				\item POI
				\item POI with the hints
				\item Go around a corner with compass
				\item Go around a corner
				\item Go to a corner of two streets
				\item Reverse Geocoding
			\end{itemize}	
		\end{multicols}
		
		 
	
		\section{Describe the surroundings}
			For start I wanted to try localize the pedestrian just by, analysis of what's arround him and unprecise GPS position. I used GPS radius of 50m  and info as i.e. \uv{I am going downhill, grass area on the right hand and noisy street on the left hand.} The question was is it sufficient info? And the next question was can such solution be implemented?
			\subsection{1st iteration}
				\label{sec:surroundings-frist-iter}
				To get all necessary info from the user, I had to invent an dialog system a and dialog logic. To get inspiration my supervisor advised me to prepare an experiment.
				
				An experiment - two player game, to be precise. One player was simulating a \emph{blind user}, the second player was simulating the \emph{dialog system}. Every player was sitting on a separate table and there was an divider between them.
				
				%image two players
				\begin{figure}[th]
					\centering
					\includegraphics[width=0.7\linewidth]{../../../../Dropbox/dp/developement/surroundings/dva-hraci23062017}
					\caption{Two player game}
					\label{fig:two-players}
				\end{figure}
			
				
				\paragraph{blind user}
					This player was given a map and a marker. The marker was a piece of paper with cutout hole representing the blind person. He was then using this  marker to move on the map and see only very close surroundings. See pic \ref{fig:map-and-blind-simulator}.
					
					The map was depicting all the environment: inclination of sidewalks, pedestrian crossings,
					marking for blinds, rails, houses, parked cars, noises form cars, ticking noises from traffic	lights, direction of cars and tram lines.
					
					The user choosed a place on a side walk he liked and put a cut out hole on that spot.
					Then he was moving and exploring the environment based
					on \emph{dialog system}'s orders and responding his questions.
				\paragraph{dialog system}
					This player had another copy of the map. He was asking the \emph{blind user} the questions and giving him orders. In order to localize his position. \uv{Do you have there buildings?}, \uv{Which hands do you have the house
					on?}, \uv{Are you going up hill?}. Depending on the response i.e \uv{I am going downhill.} He was
					eliminating all the not suitable places (uphills, and horizontal sidewalks). When the possible
					space on the sidewalks narrowed to one point the seeker anouced \uv{Found you, you are here!}
					and showed it to the \emph{blind user}.	
				
				\begin{figure}[th]
					\centering
					\includegraphics[width=0.7\linewidth]{"../../../../Dropbox/dp/developement/surroundings/map and blind simulator-01"}
					\caption{Map and marker}
					\label{fig:map-and-blind-simulator}
				\end{figure}
				
				We went with every player over 4 maps with increasing complexity. 
				I was allways playing the \emph{dialog system} player.
				
				I recorded the audio from every game session to an audiofile. I transcribed the audiofiles and then extracted all the questions and orders of the \emph{dialog system} player.
				
				Next I divided the questions to clusters according to the topic. And then selected the question, which seemed most suitable or most appropriate to me. And then put down a list of these selected questions.
			
				\begin{figure}[th]
					\centering
					\includegraphics[width=0.7\linewidth]{../../../../Dropbox/dp/developement/surroundings/selected-questions-example}
					\caption{Examle of 2 selected questions from WHERE cluster}
					\label{fig:clusteredquestions}
				\end{figure}
			
				\paragraph{testing details}
					Five friends interacted as users in this study. 2 women and 3 men. They were 21 - 53 years old. Each participant went over 4 maps.
			
			\subsection{2nd iteration}
				In this iteration I created a flow diagram in Axure\cite{axure}. 
				
				\begin{figure}[ht]
					\centering
					\includegraphics[width=\linewidth]{../../../../Dropbox/dp/developement/surroundings/2nditeration/tested-dialogue}
					\caption[]{dialogue}
					\label{fig:tested-dialogue}
				\end{figure}
			
				%\begin{figure*}[ht]
				%	\centering
				%	\begin{subfigure}[t]{0.5\textwidth}
				%		\centering
				%		\includegraphics[width=0.7\linewidth]{../../../../Dropbox/dp/developement/surroundings/2nditeration/tested-dialogue}
				%		\caption[]{dialogue}
				%		\label{fig:tested-dialogue}
				%	\end{subfigure}%
				%	~ 
				%	\begin{subfigure}[t]{0.5\textwidth}
				%		\centering
				%		\includegraphics[width=0.7\linewidth]{../../../../Dropbox/dp/developement/surroundings/2nditeration/detail}
				%		\caption[]{dialog detail}
				%		\label{fig:dialog-detail}
				%	\end{subfigure}
				%	\caption{Dialog and detail from dialogue}
				%\end{figure*}
		
				We played the same game as in Iteration 1 (see \ref{sec:surroundings-frist-iter}). With small differences:
				
				%\paragraph{blind user}
				%	Same as in 1st iteration.
				\paragraph{dialog system}
					This player was no longer talking. But he was still listening and deciding the logic. When he wanted to tell something to the \emph{blind user}, he instead only clicked on the appropriate button with pre-spoken messages. When the user replied, this player analyzed for keywords, and based on keyword he decided what button he should click on.
					
				
				The audio was done using Accapela-box text to speach service\cite{accapela}. I entered every possible text, downloaded it as TTS mp3 and linked it to correct button in Axure's flow diagram.
				
				We went over the same 4 maps with increasing complexity. I was playing the \emph{dialog system} player. But this time the maps were graphicaly improved to be easier for \emph{blind user} player.
				
				I did usability analysis of the sessions and discovered 27 usability issues. The major one was \emph{blind player} had no idea what does it mean \uv{Continue further in jour journey.}. The instruction is \uv{continue further until you find something interesting}. He is going through the street and reaches a corner, report it to the system and the system answers \uv{Ok, you are at a corner. Now continue further}. Because it can have different meanings in the following situation. Should the user turn on the corner or cross the street?
						
		
				%img of the corner	
				\begin{figure}[th]
					\centering
					\includegraphics[width=0.3\linewidth]{../../../../Dropbox/dp/developement/surroundings/2nditeration/CCI05042016}
					\caption{What does it mean \uv{Continue..}}
					\label{fig:what-does-continue-mean}
				\end{figure}
			
				
					
					And the next problem was the \emph{blind users} were not sure how to name some types of corners. There are many types of corners, I was not sure how the blind would recognize them.
					
				%3 images of different corner types	
				\begin{figure*}[ht]
					\centering
					\begin{subfigure}[t]{0.32\textwidth}
						\centering
						\includegraphics[width=\linewidth]{../../../../Dropbox/dp/developement/surroundings/2nditeration/ostry-s-dirou}
						\caption[]{Sharp with a space inside}
						\label{fig:ostry-s-dirou}
					\end{subfigure}%
					~ 
					\begin{subfigure}[t]{0.32\textwidth}
						\centering
						\includegraphics[width=\linewidth]{../../../../Dropbox/dp/developement/surroundings/2nditeration/kulaty}
						%\caption[]{dialog detail}
						\label{fig:kulaty}
					\end{subfigure}
					~ 
					\begin{subfigure}[t]{0.32\textwidth}
						\centering
						\includegraphics[width=\linewidth]{../../../../Dropbox/dp/developement/surroundings/2nditeration/zkoseny}
						\caption[]{dialog detail}
						\label{fig:lomeny}
					\end{subfigure}
					\caption{Example of different corners of buildings in Prague}
				\end{figure*}
					
	
			
				
				
				\paragraph{testing details}
					Five friends interacted as users in this study. 3 women and 2 men. They were 22 - 27 years old. Each participant went over 4 maps.
			\subsection{additional research}
				This research was focused to verify the ideas from dialog sytem in 2nd iteration and to get insights how to solve "continue.." issue. I In order to do that, I designed the research to answer the following questions:
					\begin{itemize}
						\item how do they call the different type of corrners
						\item how do they work with archway
						\item how do they distinguish between tactile guide linie and tactile marking at pedestrian crossing
						\item how do they recognize between regular and dropped curb
						\item how do they work with grass area
						\item how do they work with railings
						\item what they can say about tramlines
						\item how do they describe tilt of sidewalks
					\end{itemize}
				
				The research showed:
					They will often not report the corner as the correct type. Because they don't touch the corners just try to walk close to the wall. And then report the shape of the corner based on how they walked and how they felt the wall was close to them during the whole maneuver.
					They don't register global changes, only local i.e. long slow inclination will go unnoticed. Corner less than $45\deg$ can be unnoticed as well.
					They register only local significant changes.
					They can not notice the tactile marking at pedestrian crossing.
					They use the grass as signal linie and call it just "grass".
					Trams are very good indicie. They go on Charles Square frequently.
					The streets are very noisy. Sometimes you don't hear the other person speaking. Not always suitable for dictating.
					
					
					%corners and how they walk arround
					\begin{figure}[th]
						\centering
						\includegraphics[width=0.7\linewidth]{../../../../Dropbox/dp/developement/surroundings/corners-research}
						\caption{Four types of corners and the red line demonstrates, how blind people walk around the corner. There is a similarity in path around $90\deg$ and small rounded corner. and another similarity is with walking around the large rounded corner and large bevel corner.}
						\label{fig:corners-research}
					\end{figure}
				
				\paragraph{testing details}
					Three blind people interacted as users in this study. 3 men. They were 21 - 64 years old. Each of the three participant walked through the same commented path.
			\subsection{3rd iteration}
				In this iteration I fixed the flaws from second iteration and used the visdom from the additional research.
				I fixed:
					asking about how long he walked from the last reported point
					asking how he went (crossing street, go left arround corner, gor right)
					reducing the types of corners to round, sharp and $45\deg$ and reading the list of choices, in case the user called the corner differently
					and changing some wording.
									
				
				\begin{figure}[th]
					\centering
					\includegraphics[width=0.7\linewidth]{"../../../../Dropbox/dp/developement/surroundings/3rd iteration/dialog - testovany prototyp"}
					\caption{}
					\label{fig:dialog---testovany-prototyp}
				\end{figure}
			
				I tested with three users in the areas close to Charles' square. But the testing was successfull only in 5 cases out of total 12 tries. It failed mostly, because the user described the surroundings differently then the system was prepared to process and response. i.e. S:\uv{What kind of corner?} U:\uv{I am on a crossing}. Or the system discovered the user is somewhere on a corner of a long segment of sidewalk, but was not able to determine which corner was it.
				
				Before evaluating this testing session  more deeply, tried to discover what info exactly is in the Naviterier system, and had to cancel this method. See following section \emph{Canceled}. therefore it didn't made sense to spent more time on analysing this testing session.
				
				\paragraph{testing details}
					Three friends interacted as users in this study. 2 women and 1 men. They were 22 - 26 years old. Each participant went over 4 places in the city.		
			\subsection{Canceled}
				Here comed the hacking and reverse engineering. Because the Naviterier API \cite{naviterier-api} don't return info about the types of corners. I used the Naviterier's route description API \cite{naviterier-route-description} to find the stored type of corners. I allways pick one address of a house in front of the corner and one address of a house just behind the corner and let the API to generate the description. In the description was then mentioned the type of the corner, so I was able to discover, what kind of info the future version of the Naviterier's API\cite{naviterier-api} can provide.
				
				I discovered the method \emph{describe surroundings} suffers from two major troubles. The first the system can't rely the user will describe the environment as it's  described in the map. The second the records in the map are sometimes not corresponding fully with reality.
				
				The first, hits into classic contrast between recognition and recall. i.e. in the map is the description the sidewalk is going slowly uphill. then the blind people will pay attention and confirm the system is right. On the other hand in this prototype, we ask users to describe the inclination of the sidewalk. And the research discovered, blind people sometimes don't notice if the inclination of the sidewalk is very slow. I would see it as the sighted person, because i see more of the area and analyse visualy a long piece of the street. In contrast to that the blinds only analyse a short part of the sidewalk and can report mild inclination as flat - no inclination. 
				
				To corners, the blind people don't touch corners. If you say them, you are on a round corner, they just think \uv{yeah, can be.}
				
				For the second, again recognition vs. recall. In Naviterier map database\cite{naviterier-map-details} is stored informations, which are sufficient for task: \emph{System describes the surroundings to user, and user will agree he is right}, but this info is not sufficient for reverse direction task \emph{User describes the surroundings, and the system says you are exactly here.} i.e. the corner on picture \ref{fig:ostry-s-dirou} is in the Naviterier's DB stored as \emph{bevel corner}. While the user's in the research described it as \emph{right angle}, \emph{right angle with a space inside} or just as an \emph{entrace to the building}.
				
				Or the corner on the picture \ref{fig:lomeny} was dascribed as \emph{two times $45\deg$} or as \emph{rounded corner}, while the Navitrier's DB is \emph{street is bending to the left}.
				
				Because of these two major issues, I decided to drop this approach and design another methods.
				
		\section{POI}		
			\subsection{1st iteration}
			\subsection{2nd iteration}		
		\section{POI with the hints}
			\subsection{1st iteration}
			\subsection{Canceled}	
		\section{Go around a corner with compass}
			\subsection{1st iteration}
			\subsection{2nd iteration}
		\section{Go around a corner}
			\subsection{1st iteration}
			\subsection{Canceled}
		\section{Go to a corner of two streets}
			\subsection{1st iteration}
			\subsection{Canceled}
		\section{Reverse Geocoding}
			\subsection{1st iteration}
			\subsection{Canceled}
		
		
	\chapter{Proposed Methods}
		This section describes all methods I found out. The description talks allways about the latest iteration of the prototype.
		\section{Overview}
		The iterative design process lead me to propose and validate 7 different methods. These methods are split into categories without GPS and without. The first group, without GPS promises working even on old phones as Nokias with Symbian S60, which are still used by some blind. The second group with GPS uses the advatages of modern phones as iPhones and even recent cheap Androids, with the basic smart sensors.
		
		\begin{figure}[ht]
			\centering
			\includegraphics[width=0.8\linewidth]{../../../../Dropbox/dp/overview/prototypes_testings}
			\caption{Overview of which prototype survived to which testing}
			\label{fig:prototypestestings}
		\end{figure}
		
		\section{List of all prototypes}
		\paragraph{Without GPS}
		\begin{description}
			\item [POI]
			The systems asks questions about the current address or current start stop name to decide the users exact position.
			\item [Go to corner of two streets]
			The system asks user to go to a crossing of two streets and asks pedestrians walking arround to get the name of the streets. 
		\end{description}
		\paragraph{With GPS}
		\begin{description}
			\item [Describe surroundings]
			User describes, what's arround him i.e. \uv{road on the left, grass area on left} and the system looks on the map in user's area and asks follow up questions until, the descriptions match exactly one position on exactly one sidewalk.								
			\item [POI with hints]
			Similiar as the POI without GPS. But when asking about the address or tramstop, it tells the user the streets and stop's names which are nearby based on GPS signal.
			\item [Go arround corner]
			Uses GPS position of target and compass in the phone, to tell user what direction he should start. Then leads the user arround corner while collecting the GPS positions. And then finaly estimate the user's sidewalk by matching of collected GPS coordinates sidewalks in the map.
			\item [Go arround corner without compass]
			Same as the previous, but on the beginning to decide for himself, what direction he wants to start.
			\item [Reverse geocoding]
			Uses GPS position to get coordinates, gets current address from them and then launch navigation between the current address and the target address.
		\end{description}
		
		
		
		\section{Without GPS}
		\section{GPS}
		\subsection{Go arround corner with compass}
		\begin{figure}[!htb]
			\minipage{0.32\textwidth}
			\includegraphics[width=\linewidth]{../../../../Dropbox/dp/proposed-methods/reverse-gc/01}
			\caption{A really Awesome Image}\label{fig:awesome_image1}
			\endminipage\hfill
			\minipage{0.32\textwidth}
			\includegraphics[width=\linewidth]{../../../../Dropbox/dp/proposed-methods/reverse-gc/02}
			\caption{A really Awesome Image}\label{fig:awesome_image2}
			\endminipage\hfill
			\minipage{0.32\textwidth}%
			\includegraphics[width=\linewidth]{../../../../Dropbox/dp/proposed-methods/reverse-gc/03}
			\caption{A really Awesome Image}\label{fig:awesome_image3}
			\endminipage
		\end{figure}
		
		This prototype uses the compass to send the user the correct way, then it collects GPS coordinates on the way
		This is a high fidelity prototype.
		\subsection{Reverse geocoding}
		
		\begin{figure}[!htb]
			\minipage{0.32\textwidth}
			\includegraphics[width=\linewidth]{../../../../Dropbox/dp/proposed-methods/reverse-gc/01}
		%	\caption{A really Awesome Image}\label{fig:awesome_image1}
			\endminipage\hfill
			\minipage{0.32\textwidth}
			\includegraphics[width=\linewidth]{../../../../Dropbox/dp/proposed-methods/reverse-gc/02}
		%	\caption{A really Awesome Image}\label{fig:awesome_image2}
			\endminipage\hfill
			\minipage{0.32\textwidth}%
			\includegraphics[width=\linewidth]{../../../../Dropbox/dp/proposed-methods/reverse-gc/03}
			%\caption{A really Awesome Image}\label{fig:awesome_image3}
			\endminipage
		\end{figure}
		
		Reverse geocding method uses GPS sensor only one time to get actual GPS coordinates. 
		The system asks the user for the target adress. Meanwhile it launches the GPS sensor.
		The user is typing the desired target address on default system keyboard or in Chrome browser, he can use the advantage of button \uv{Dictate} and speak it aloud.
		When the user is done with input, the system shows current estimated accuracy of the system.
		Then user can click on \uv{Navigate to Vodičkova 13}. The value of this button changes according to the inputed target address.
		Once the user click navigate, the systems validate the target address and the user location are both in area covered by Naviterier and if so the system shows the user the navigation. Which then navigate the user to the target. See section \ref{sec:navifation} for more details.
		
		Note: The estimated accuracy is changed only when the sensor tells a value which is $\geq 5m$. This prevents the screen reader from infinite reeding loop, when it would be changing between i.e. $10m$ and $11m$. 
		\section{Navigation}
		\label{sec:navifation}
		This describe the navigation itself. It means this decribe the window which is opened when the system estimates the exact user address.

		
		
				
	
	
	

	\chapter{Software Implementation}
		For the last 5th experiment I implemented the prototypes in  Django framework \cite{django}.
		\section{Common}
			In this section I describe the implementation part common to all 3 Hi-Fi implemented prototypes.
		
		
		\subsection{POI}
			TODO		
		\subsection{Go around a corner}
			TODO
		\subsection{Reverse Geocoding}
			TODO
			
		\subsection{Other prototypes}
			No implementation parts.
			\emph{Go to a corner of two streets}, 
			\emph{Describe the surroundings},
			\emph{POI with the hints},
			\emph{Go around a corner without the compass}
			 were realized only as a wizzard of ozz. With no real coding. I only drawn a flow diagram in Axure and enriched it with prerecorded audio mp3s from Accapela Text to speach service.
	\chapter{Testing}
		\section{POI}		
		\section{Go to a corner of two streets}		
		\section{Describe the surroundings}
			\subsection{1st iteration}
				\begin{itemize}
					\item at home over map
					\item 
				\end{itemize}
			\subsection{2nd iteration}			
			\subsection{3rd iteration}	
		\subsection{Canceled}		
		\section{POI with the hints}
		\section{Go around a corner}
		\section{Go around a corner without the compass}				
		\subsection{Reverse Geocoding}
			Google gecoding doesn't allways return address
		
		
						
		
		\section{Testing sessions}
			\subsection{1st testing}
				\label{sec:1th-testing}
			\subsection{2nd testing}
				\label{sec:2th-testing}
			\subsection{3rd testing}
				\label{sec:3th-testing}
			\subsection{4th testing}
				\label{sec:4th-testing}
			\subsection{5th testing}
				\label{sec:5th-testing}
		\section{Research sessions}
			\subsection{what they recognize on streets}
			\subsection{names of streets}
			\subsection{how they get lost}
			\subsection{how they write}
	
	\chapter{Results}
		\section{Prototypes without GPS}
			\subsection{POI}		
			\subsection{Go to a corner of two streets}					
			
		\section{Prototypes with GPS}
			\subsection{Describe the surroundings}
		
			\subsection{POI with the hints}
			\subsection{Go around a corner}
			I see future in this approach, guide the person using the compass, and when sure, where he is, switch to precise Naviterier navigation.
			The ease of use is big.
			The UI should be improved by saying which direction he should start just one time. It current version the command is allways changing. So when the user turns himself in desired position, he turns the phone as well, and the phone then can tell different direction than before. Which leads to confused spinning on a place. One possible fix can be \uv{your target is on 3 o'clock, start walking along the buildings, that direction}
			The algorithm, should be improved in two ways. By asking the user on which hand he has the houses.
			\subsection{Go around a corner without a compass}
				
			\subsection{Reverse geocoding}
				I don't recomend this method for future usage. This method estimate the correct sidewalk with 50\% success. It can tell the correct sidewalk, but sometimes it either says the user ison the sidewalk on the opposite side of the street.  And thats huge difference. 
				Because after deciding the correct sidewalk/address, the phone downloads the way itinereary and turn off gps. It's a huge difference if the itinerary says, \uv{stand with your backs to wall and go left} or if it says \uv{...and go right} while giving the orders for the sidewalk accross the street. Therefore the whole navigation is then random and can't be used at all.
				
				The used reverse GC API was not working 100\%. The API has to be replaced. See section \ref{subsec:reverse-gc-api} for more details.
							
				This method can't be used as a production method for start of the navigation system.
				This method can be used only as quick comparsion for experiments (with fixed reverse GC API).
		
		\section{Other parts of the solution}	
			\subsection{Reverse geolocation API}
				\label{subsec:reverse-gc-api}
				Don't use the Google API or HERE API. Because Naviterier requires Street name and house number as na input i.e. \uv{Vodičkova 13}. But both of them sometimes, instead of the address  returns just the street name i.e. \uv{Vodičkova}, or just the area name ie \uv{Nové Město, Praha}. Therefore you need an API which allways returns Street and housenumber. CEDA API seems promising. Still better method would be an API which for GPS coordinates returns the sidewalk Id and don't rely, that the nearest address entery point is on the same sidewalk.	
			\subsection{Naviterier}
				The navigation is very precise and appreciated by the blinds. However they don't know it's just pre-downloaded set of orders. That it does not update base on the real location.
				
				I definitely recommend either make sure the user knows the navigation started to work off-line or add the updating based on real-time GPS signal. In the tested systems when the system used GPS to locate the initial position, and then turned off the signal without noticing the user, it made the users make huge mistakes, guiding them the wrong path, while still they were sure, they are walking the right way and the system as well knows their exact location.
			
	
	\chapter{Appendix}
		\section{POI bugs}
			The testing described in section \ref{sec:5th-testing} discovered the positions of some tram-stops are wrong. The difference can be up to $40m$. On the picture \ref{fig:myslikova-spatne-tram} the participant crossed the street within the pedestrian crossing and ended up in street \emph{Černá} instead of \emph{Myslíkova}. While thinking he is still correct.
			
			The positions of tram-stops has to be verified and fixed.
			\begin{figure}[!htb]
				\minipage{0.49\textwidth}
				\includegraphics[width=\linewidth]{"../../../../Dropbox/dp/chyby/mhd/lazarska-spatne-tram"}
				\caption{Badly placed Lazarská stop. In circle is the position in MHD database, the cross is the real position of the stop}\label{fig:awesome_image1}
				\endminipage\hfill
				\minipage{0.49\textwidth}
				\includegraphics[width=\linewidth]{"../../../../Dropbox/dp/chyby/mhd/myslikova-spatne-tram"}
				\caption{Badly placed Myslíkova stop. Same marking as previous image.}\label{fig:myslikova-spatne-tram}
				\endminipage\hfill
			\end{figure}
		
		
	
	

	
	
	
	%*****************************************************************************
	% Seznam literatury je v samostatnem souboru reference.bib. Ten
	% upravte dle vlastnich potreb, potom zpracujte (a do textu
	% zapracujte) pomoci prikazu bibtex a nasledne pdflatex (nebo
	% latex). Druhy z nich alespon 2x, aby se poresily odkazy.
	
	\bibliographystyle{abbrv}	
	%bibliographystyle{plain}
	%\bibliographystyle{psc}
	{
		%JZ: 11.12.2008 Kdo chce mit v techto ukazkovych odkazech take odkaz na CSTeX:
		\def\CS{$\cal C\kern-0.1667em\lower.5ex\hbox{$\cal S$}\kern-0.075em $}
		\bibliography{reference}
	}
	
	% M. Dušek radi:
	%\bibliographystyle{alpha}
	% kdy citace ma tvar [AutorRok] (napriklad [Cook97]). Sice to asi neni  podle ceske normy (BTW BibTeX stejne neodpovida ceske norme), ale je to nejprehlednejsi.
	% 3.5.2009 JZ polemizuje: BibTeX neobvinujte, napiste a poskytnete nam styl (.bst) splnujici citacni normu CSN/ISO.
	
	%\include{appendix}
	
\end{document}

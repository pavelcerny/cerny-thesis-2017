TAGLINE - jakou metodu vybrat pro lokalizaci slepce, když chce pak někam dojít
	\chapter{Analysis}
	The GPS system can locate you with precision up to 1 meter. That's true just outside the city in areas with clean view to the sky. In center of Prague (Czech Republic). It's not so perfect. The city area, covered with houses and narrow streets. The precission differs.
	This is the recording of a GPS:
	%image A showing walked history
	
	CAPTION A - this is what was recorded during the walk
	% image B showing real path
	CAPTION B - ...and this is the path the user really took.
	
	There is a siginificant difference. And we can clearly say the GPS in Prague is not sufficient to determine which side of the street we are standing on.
	%image demonstrate what is left and what right sidewalk
	 And some times even can't say, what street are we standing in.
	% image gps in the middle of a house
	
	\section{GPS accuracy}
	I performed a test in order to know how acurate is the GPS in the city. I used two phones to collect the data. Both of them were running app GPS logger.
	
	phone \#1  BlackBerry Q10
	phone \#2  Android - HTC Desire X
	
	app		GPS logger \ref{gps-logger}
	
	I walked 1.1km long path arround Charles Square and nearby surroudings. The app logged the GPS postion on both phones every 60 seconds of walk. This way I collected 23 recordings (totall). Then I compared these recordings with real positions on the map. And I computed the difference between estiamted accuracy and real accuracy.
	
	%img tables - přesnost GPS z SVP
	
	The results showed two things:
	\begin{enumerate}
		\item the real accuracy is nearly allways $\leq$ estimated accuracy
		\item the average estimated accuracy is mostly allways $\leq 50m$
	\end{enumerate}
	
	
	
	
	\section{Area of Interest}
	
	\section{Naviterier}
	
	\section{Goals}
		\begin{itemize}
			\item dialog application
			\item without GPS
			\item \emph{ (or as minimally as possible)}
		\end{itemize}